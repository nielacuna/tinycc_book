\documentclass[11pt, oneside]{book}              % Book class in 11 points
\usepackage[T1]{fontenc}
\parindent0pt  \parskip10pt             % make block paragraphs
\raggedright                            % do not right justify

% use \verbatim{} instead of lstlisting.
%\usepackage{listings}
%\lstset{
%        keepspaces=true,
%        numbers=left,
%        numbersep=10pt,
%        language=C,
%}


% use this when embedding codes
\newenvironment{tcc_desc}
        {\everypar{\hangafter=1 \setlength{\hangindent}{3em}}} % begin-code
        {\par}  % end-code


\title{\bf The Tiny C Compiler by Fabrice Bellard}    % Supply information
\author{Niel Anthony E Acuna}              %   for the title page.
\date{\today}                           %   Use current date. 


\includeonly{
        introduction,
        memory,
        tokens,
        symbols,
        string,
        dynarrays,
        preprocess,
        lexer,
}

% Note that book class by default is formatted to be printed back-to-back.
\begin{document}                        % End of preamble, start of text.
\frontmatter                            % only in book class (roman page #s)
\maketitle                              % Print title page.
\tableofcontents                        % Print table of contents




\mainmatter                             % only in book class (arabic page #s)


\part{Introduction}                   % Print a "part" heading

\chapter{Introduction}

\section{Getting Started}

git clone


\section{Configuration and Building}

./configure
make
make install

\section{Source of Information}

\section{Browsing the Code}


exuberant ctags
will help with most of the symbols and functions, but TCC uses some tricks 
when generating defines that the ctags parser can not reseaonably pick up on. This is just how things 
are done inside the TCC code base and when encountering cases like the one mentioned above,
one can turn to some good 'ol code sleuthing using tools like find and grep and sometimes, 
just painfully comb through each one of the source codes looking for the answer.

eclipse-cdt



\part{Tiny C Compiler}

\chapter{Memory Management}
\section{System Allocator}
\section{Tiny Allocator}



\chapter{String Management}


\section{The CString Object}

CStrings are TCC implementations of raw C strings (character arrays). Because 
Raw C strings are usually tags or pointers to contiguous regions of memory,
dynamically resizing them (more commonly, enlarging them) is tedious. A new
larger memory block needs to be allocated and the old string copied to this new
block. Care must be taken so that the new larger block of memory is large enough
for the new additions to the string to be added safely. Also, the old memory
address of the string must be \verb|free()|'d back to the platform so it can be reused.

It is declared as follows in <tcc.h>.
\begin{verbatim}
425 typedef struct CString {
426     int size;
427     void *data;
428     int size_allocated;
429 } CString;
\end{verbatim}

\begin{tcc_desc}
\textbf{size} The current memory usage of the CString object - which is also the current string length not counting the terminating NULL. This must always be less than or equal to the \verb|size_allocated| member field.

\textbf{data} This is a pointer to the memory region containing the raw C string.

\textbf{size\_allocated} Describes the maximum growable size of the CString. This is not the actual size of the string, that is what the \verb|size| member field is for. The CString is usually reallocated to a bigger memory when it is determined that future concatenations to the string will overflow \verb|size_allocated| soon.
\end{tcc_desc}

\subsection{Creating CStrings}

\begin{verbatim}
370 ST_FUNC void cstr_new(CString *cstr)
371 {
372     memset(cstr, 0, sizeof(CString));
373 }
\end{verbatim}
CString objects start their life as an empty string. They don't hold any strings at all and subsequently have a length of 0. This function ensures the CString is in a known state before even the first string is assigned to it.

\subsection{Destroying CStrings}

\begin{verbatim}
ST_FUNC void cstr_free(CString *cstr)
{
    tcc_free(cstr->data);
    cstr_new(cstr);
}
\end{verbatim}
free string and reset it to NULL.

\subsection{Resetting CStrings}

*** ST\_FUNC void cstr\_reset(CString *cstr)
reset string to empty.

\subsection{Concatenating a single Character}

*** ST\_INLN void cstr\_ccat(CString *cstr, int ch)
add a byte. reallocate memory if needed.

\subsection{Concatenating a string to a CString}

*** ST\_FUNC void cstr\_cat(CString *cstr, const char *str, int len)
add a range of bytes, range is defined by len parameter

\subsection{Formatted Concatenation to CStrings}

*** ST\_FUNC int cstr\_printf(CString *cstr, const char *fmt, ...)
provides a way to continuously append strings to a CString.


ST\_FUNC void parse\_mult\_str (CString *astr, const char *msg)


\section{Other String Handling Patterns}

tcc\_split\_path()



/* copy a string and truncate it. */
ST\_FUNC char *pstrcpy(char *buf, size\_t buf\_size, const char *s)


/* strcat and truncate. */
ST\_FUNC char *pstrcat(char *buf, size\_t buf\_size, const char *s)


ST\_FUNC char *pstrncpy(char *out, const char *in, size\_t num)



\chapter{Dynamic Arrays}
TCC makes extensive use of dynamic arrays. arrays that can regrow itself 
so it can accomodate more entries.

example of dynarrays:
        state->include\_paths
        state->sysinclude\_paths
        state->library\_paths
        state->crt\_paths
        state->files => dynamic array of filespec objects



\chapter{Filename Processing}

\chapter{BufferedFile}

\chapter{The TCC State Object}

\chapter{Error Processing}

\part{Lexical Analysis}

In compiler parlance, lexical analysis is the process of breaking down a sequence of characters or in our case, a C source code into a collection of basic language elements called tokens.


\chapter{Tokens and Lexemes}

A token is the most basic element that comprises a language. It is an indivisible unit lexical unit. For example, in the C language, keywords like while and for are tokens. Symbols like \emph{>} or \emph{++} or \emph{>=} or \emph{>>} are also tokens. Identifiers such as variable names, function names are tokens. The original string that comprises a token is called a \emph{lexeme}. There is not a one-to-one relationship between lexemes and tokens. A name or number token for example can have many possible lexemes associated with it. C keyword tokens like \emph{while} or \emph{for} or \emph{define} always matches a single lexeme. 

\section{Tokens in TCC}

\subsection{TokenSyms}

TokenSyms is how TinyCC represents the tokens internally. When the lexer encounters a character sequence it thinks can stand as a token like a function identifier or a C keyword or a macro identifier, it will create a TokenSym for that character sequence if it still hasn't.

There are two kinds of tokens in TinyCC, static tokens, the strings of which are compiled into TinyCC itself and dynamic tokens, which are tokens created by the lexer while it is scanning through the source code being compiled.

\subsection{Static Tokens}
Static tokens are determined during compile time and are built into TinyCC. They are strings that are reserved by the C standard, or specific keywords that carries special function or meaning inside TinyCC.

The relevant listing is shown below:
\begin{verbatim}
1163 enum tcc_token {
1164     TOK_LAST = TOK_IDENT - 1
1165 #define DEF(id, str) ,id
1166 #include "tcctok.h"
1167 #undef DEF
1168 };
\end{verbatim}

\begin{tcc_desc}
\textbf{line 1164} \verb|TOK_LAST| = 255. This assures that the enumerations start at the 256 value or \verb|TOK_IDENT|. This is because the values 0-255 are all already reserved by the ASCII and ISO-8859-1 encodings and sometimes, some of these characters are returned as tokens themselves by the lexer depending on the parse flags or tok flags.

\textbf{line 1165-1166} the file \verb|tcctok.h| follows a specially formatted 2-tuple entry like \verb|DEF(TOK_INT, "int")|. The first one is used as a define constant and the second one is used as a string associated with this token. The purpose of the \verb|DEF()| is to really just filter out one of the two, depending on what part of the tuple will be used. In the case of defining token constants, the first part of the tuple is required, hence the \verb|DEF(id, str)| uses \verb|id| and leaves out \verb|str| in its definition.
\end{tcc_desc}

\subsection{More Static Tokens}

\begin{verbatim}
to be studied further... two char tokens like <=, <<, &=, ==, etc...
\end{verbatim}


\subsection{Dynamic Tokens}

While C language reserved keywords like int, void, inline and etc have their values computed statically upon building TinyCC itself, most tokens extracted during the passes made to the source code being compiled are assigned dynamically.


\subsection{Building the Token Table}

The token table is a dynamically allocated \verb|tokensym| pointer array that can be enlarged during runtime to be able to accomodate new additional tokens as they are discovered. 

Building the table is a two step process. The first part of the table is constructed early during TCC startup. Care is taken so that the statically allocated tokens are placed in the table with their token identifiers intact - that is, the token numbers that were assigned for them during building of the TinyCC compiler itself are not changed, relative to \verb|TOK_IDENT|. This means those same static tokens have constant token numberings, e.g. \verb|TOK_INIT| will always be 256. The code is elaborated inside the \verb|tccpp_new()| function below.

\begin{verbatim}
3667     tok_ident = TOK_IDENT;
3668     p = tcc_keywords;
3669     while (*p) {
3670         r = p;
3671         for(;;) {
3672             c = *r++;
3673             if (c == '\0')
3674                 break;
3675         }
3676         tok_alloc(p, r - p - 1);
3677         p = r;
3678     }
\end{verbatim}

\begin{tcc_desc}
\textbf{line 3667} set the global \verb|tok_ident| variable to start at \verb|TOK_IDENT| which is the value 256.

\textbf{line 3668} \verb|p| is always used as a pointer to the first letter of a substring.

\textbf{line 3669} exhaust all characters inside \verb|tcc_keywords| array.

\textbf{line 3672-3675} \verb|r| is a pointer to the terminating \verb|NULL| of the current substring. A simple pointer subtraction \verb|r - p - 1| will yield the substring length.

\textbf{line 3676} create a token entry now in the token table and also add it to the hashtable. Passing the substring (lexeme) and the substring length as parameters.

\textbf{3677} \verb|p| points to next substring in the sequence now.

\end{tcc_desc}

Each token is ofcourse associated with a string. Let's inspect how \verb|tcc_keywords| is created at \verb|tccpp.c| below.

\begin{verbatim}
59 static const char tcc_keywords[] = 
60 #define DEF(id, str) str "\0"
61 #include "tcctok.h"
62 #undef DEF
63 ;
\end{verbatim}
\begin{tcc_desc}
\textbf{line 59} \verb|tcc_keywords| is a constant character array.

\textbf{line 60-61} create a function-like macro called \verb|DEF()| to filter out the string part of the token-string tuple defined in \verb|tcctok.h|. Additionally, append a terminating \verb|NULL| character to the string. This creates a very long \verb|tcc_keywords| string containing substrings inside that are separated by the \verb|NULL| character.

\textbf{62} \verb|DEF()| macro's purpose has been fulfilled, undefine it so it can be reused in the future.
\end{tcc_desc}


\subsection{TokenStrings and Management of TokenStrings}

Used to record constant values and function macroses.

\subsubsection{Creation and Initialization of a TokenString}

TokenStrings are always created empty and simply just enlarged during runtime as the need arises.

\subsubsection{Duplicating a TokenString}

\subsubsection{Destroying a TokenString}

\subsubsection{Growing the string buffer inside a TokenString Dynamically}

\subsubsection{Adding a new token to a TokenString}


\subsubsection{Adding new Constant to a TokenString}

\subsection{TokenString Macros}

\subsubsection{Beginning a Macro}

\subsubsection{Ending a Macro}



\subsection{Adding new tokens to the system}




\subsubsection{The Token Table}

The token table is the actual repository of all the tokens being used by the compilation environment when processing a source file.

\subsubsection{The Token Hash}

The token hash is used for fast searching of tokens. It is important to be able to search quickly for dupicate tokens being used, like for example, if the source code being compiled has two defines for a macro, then it should be detected by TCC and the user should be alerted of the fact.

There are a few handy APIs to use when adding new tokens to the system. A token found in the token table also has an entry in the token hash.

\subsubsection{token\_alloc}

The \verb|tok_alloc()| function adds layers of processing when creating a new token from a string. It first searches the token hash if such a string is already associated with a token and if it is, no new token is created. Instead, a reference to the already existing token is returned.

On the other hand, if the string doesn't currently have any associated token, then a new token is created both in the token hash and the token table.

\begin{verbatim}
464 ST_FUNC TokenSym *tok_alloc(const char *str, int len)
465 {
466     TokenSym *ts, **pts;
467     int i;
468     unsigned int h;
469     
470     h = TOK_HASH_INIT;
471     for(i=0;i<len;i++)
472         h = TOK_HASH_FUNC(h, ((unsigned char *)str)[i]);
473     h &= (TOK_HASH_SIZE - 1);
474 
475     pts = &hash_ident[h];
476     for(;;) {
477         ts = *pts;
478         if (!ts)
479             break;
480         if (ts->len == len && !memcmp(ts->str, str, len))
481             return ts;
482         pts = &(ts->hash_next);
483     }
484     return tok_alloc_new(pts, str, len);
485 }
\end{verbatim}

\begin{tcc_desc}
\textbf{line 470-475} Compute the hash index for the string. set \verb|pts| to point to the slot index where this new token should go to.

\textbf{line 476-483} Do a search in the hash index for this token, taking into consideration hash collisions. When a token collides with other tokens on the same index, the list turns into a singly linked list.

\textbf{line 477} Dereference the tokensym pointer in current hash index pointer.

\textbf{line 478-479} If we have reach an empty index OR the end of the linked list (when there is a hash collision), we know that it is our first encounter with this string and no token for it has been created yet. Move to line 484 now to create a token for the string.

\textbf{line 480-481} We are looking at a potential token object containing the same string, and if current token indeed does have the same string, we already have created a token for string so we don't need to create a new one. Return a reference to the token instead.

\textbf{line 482} Hash collisions degrade the search functionality of a hash into a singly linked list. Simply move to the next token in this linked list.

\textbf{line 484} Create a new token for string and put a reference to this new token on the hash table slot pointed to by \verb|pts|.
\end{tcc_desc}

\subsubsection{tok\_alloc\_new}

The \verb|tok_alloc_new| function is a lowlevel API that requires a token hash slot as a parameter where it will save a reference to the new token it will create.

\begin{verbatim}
429 static TokenSym *tok_alloc_new(TokenSym **pts, const char *str, int len)
430 {
431     TokenSym *ts, **ptable;
432     int i;
433
434     if (tok_ident >= SYM_FIRST_ANOM) 
435         tcc_error("memory full (symbols)");
436
437     /* expand token table if needed */
438     i = tok_ident - TOK_IDENT;
439     if ((i % TOK_ALLOC_INCR) == 0) {
440         ptable = tcc_realloc(table_ident, (i + TOK_ALLOC_INCR) * sizeof(TokenSym *));
441         table_ident = ptable;
442     }
443
444     ts = tal_realloc(toksym_alloc, 0, sizeof(TokenSym) + len);
445     table_ident[i] = ts;
446     ts->tok = tok_ident++;
447     ts->sym_define = NULL;
448     ts->sym_label = NULL;
449     ts->sym_struct = NULL;
450     ts->sym_identifier = NULL;
451     ts->len = len;
452     ts->hash_next = NULL;
453     memcpy(ts->str, str, len);
454     ts->str[len] = '\0';
455     *pts = ts;
456     return ts;
457 }
\end{verbatim}

\begin{tcc_desc}
\textbf{line 434-435} TCC has exhausted the token id space and cannot give out tokens anymore. This is a fatal error.

\textbf{line 438} Normalize the token identifier so we can insert it into the token table. Token numbering starts out with \verb|TOK_IDENT| which is 256, but in reference to the token table, this is index 0.

\textbf{line 439} We trigger token table memory expansion every 512th instance.

\textbf{line 440} Grow the token table by +512 slots everytime we need to expand it. We only need to expand it by 512, everytime we fill up all the new 512 slots we added to the table at a prior reallocation.

\textbf{line 441} Update the \verb|table_ident| pointer to new bigger array.

\textbf{line 444-454} Allocate a new tokensym object with enough space for the string. Initialize this new token with its assigned token number and the string associated with it.

\textbf{line 456} Save a reference of this new token to its proper slot in the token hash.
\end{tcc_desc}

\subsection{Global Variables}

\subsubsection{tokc}
\verb|tokc| holds the lexeme of the most recently extracted constant value.

\subsubsection{tok\_flags}

\begin{verbatim}
#define TOK_FLAG_BOL   0x0001
#define TOK_FLAG_BOF   0x0002
#define TOK_FLAG_ENDIF 0x0004
#define TOK_FLAG_EOF   0x0008
\end{verbatim}


\subsubsection{parse\_flags}

\begin{verbatim}
#define PARSE_FLAG_PREPROCESS 0x0001
#define PARSE_FLAG_TOK_NUM    0x0002
#define PARSE_FLAG_LINEFEED   0x0004
#define PARSE_FLAG_ASM_FILE   0x0008
#define PARSE_FLAG_SPACES     0x0010
#define PARSE_FLAG_ACCEPT_STRAYS 0x0020
#define PARSE_FLAG_TOK_STR    0x0040
\end{verbatim}

\subsubsection{BufferedFile *file}

\subsubsection{ch and tok}

\subsubsection{tokc}

\subsubsection{macro\_ptr}

\subsubsection{tok\_ident}

\section{How it all fits in}

Given a lexeme (or identifier) in a source code, it is important to find the dynamically assigned token numbering that was associated with it during parsing. It is this token number that will connect us to all the information that was gleaned for this identifier. For example, is this identifier a macro? or a global variable? or a function? 

The token number assigned for this identifier is found in the TokenSym object that was created for it. 

How do we find the TokenSym for this lexeme? Well, we compute a hash bucket index, and find the TokenSym in the token hash table.

The \verb|tok| member of the \verb|TokenSym| structure is what we need. We take this \verb|tok| value and normalize it by subtracting it with \verb|TOK_IDENT| so we can have a valid \verb|table_ident| index.

Now that we have the \verb|TokenSym| object for this identifier (lexeme), we now have access to the actual symbol that will actually give it more context - e.g. if this identifier was defined as an macro-object, what constant value does it hold? is this contant value an int? or a float? This is all recorded in the \verb|Sym| that was also created for this identifier.


\chapter{Lexer}

%parse_flags = PARSE_FLAG_PREPROCESS | PARSE_FLAG_TOK_NUM | PARSE_FLAG_TOK_STR;

% TOK_FLAG_BOL in combination with PARSE_FLAG_PREPROCESS triggers the preprocessor
% when it encounters the # sign.

% PARSE_FLAG_TOK_NUM returns numbers instead of TOK_PPNUM

% PARSE_FLAG_TOK_NUM returns linefeed is returned as a token.



\part{Parsing}


\chapter{Symbols}
Symbols are TinyCC representations of various entities like variable names, function names and other objects.


\section{The symbol pool}
Symbols are allocated from a symbol pool. And the symbol pool is inside a dynamically enlarged collection of symbol pools. The function \verb|__sym_malloc()| takes care of managing this collection. It is explained further below.

\begin{verbatim}
tccgen.c
40 static Sym *sym_free_first;
41 static void **sym_pools;
42 static int nb_sym_pools;
\end{verbatim}
\begin{tcc_desc}
\textbf{line 40} The next sym object to be given away.

\textbf{line 41} Collection of sym pools array.

\textbf{line 42} Count of pools in the sym pools array.
\end{tcc_desc}

\subsection{Allocating/Enlarging the symbol pool}

\begin{verbatim}
691 static Sym *__sym_malloc(void)
692 {
693     Sym *sym_pool, *sym, *last_sym;
694     int i;
695 
696     sym_pool = tcc_malloc(SYM_POOL_NB * sizeof(Sym));
697     dynarray_add(&sym_pools, &nb_sym_pools, sym_pool);
698
699     last_sym = sym_free_first;
700     sym = sym_pool;
701     for(i = 0; i < SYM_POOL_NB; i++) {
702         sym->next = last_sym;
703         last_sym = sym;
704         sym++;
705     }
706     sym_free_first = last_sym;
707     return last_sym;
708 }
\end{verbatim}

\begin{tcc_desc}
\textbf{line 696} Allocate a pool of symbols.

\textbf{line 697} Add this new pool to the sym pools array.

\textbf{line 699} Save a reference to the last sym on the current sym pool.

\textbf{line 700-705} Link all the symbols in this new sym pool together. This is a backwards pointing singly linked list of free sym objects.

\textbf{line 706-707} Return a reference to the first (actually the last) sym object in this pool.
\end{tcc_desc}

\subsection{Allocating a symbol}
Symbols are allocated from a pre-allocated pool of syms. The next free sym to give out is simply the next sym object in a singly linked list of free syms as shown below inside the \verb|sym_malloc()|.


\begin{verbatim}
710 static inline Sym *sym_malloc(void)
711 {
712     Sym *sym;
713 #ifndef SYM_DEBUG
714     sym = sym_free_first;
715     if (!sym)
716         sym = __sym_malloc();
717     sym_free_first = sym->next;
718     return sym;
719 #else
720     sym = tcc_malloc(sizeof(Sym));
721     return sym;
722 #endif
723 }
\end{verbatim}
\begin{tcc_desc}
\textbf{line 714} Get a reference to the next free sym object in the current sym pool.

\textbf{line 715-716} If the current sym pool has been exhausted, allocate a new pool.

\textbf{line 717} Set up the next sym object to be given out next time a symbol is requested to be allocated.

\textbf{line 718} Return reference to this newly "allocated" symbol.
\end{tcc_desc}

\section{Freeing a symbol}

\section{Pushing a symbol}

\section{Popping a symbol}

\section{Finding a symbol}

\begin{verbatim}
736 ST_FUNC Sym *sym_push2(Sym **ps, int v, int t, int c)
737 {
738     Sym *s;
739 
740     s = sym_malloc();
741     memset(s, 0, sizeof *s);
742     s->v = v;
743     s->type.t = t;
744     s->c = c;
745     /* add in stack */
746     s->prev = *ps;
747     *ps = s;
748     return s;
749 }
\end{verbatim}

\section{Symbol Stacks}

\subsection{The define\_stack}

\subsection{The global\_stack}

\subsection{The local\_stack}

\subsection{The global\_label\_stack}

\section{The local\_label\_stack}

%\subsection{The ifdef Stack}

%\subsection{The pack Stack}





\chapter{Preprocessing}

say something about the Preprocessing stage of compilation here.

\section{The define stack}

explain what the define stack is.


The function \verb|define_push()| is used to add to the define stack. It does this by first associating a lexeme to a symbol object, then associating this symbol to a token. This symbol object can only be one of two kinds of macro types \verb|MACRO_OBJ| or \verb|MACRO_FUNC|.


\begin{verbatim}
1317 ST_INLN void define_push(int v, int macro_type, int *str, Sym *first_arg)
1318 {
1319     Sym *s, *o;
1320
1321     o = define_find(v);
1322     s = sym_push2(&define_stack, v, macro_type, 0);
1323     s->d = str;
1324     s->next = first_arg;
1325     table_ident[v - TOK_IDENT]->sym_define = s;
1326 
1327     if (o && !macro_is_equal(o->d, s->d))
1328         tcc_warning("%s redefined", get_tok_str(v, NULL));
1329 }
\end{verbatim}
\begin{tcc_desc}
\textbf{v} \verb|v| is the token for the identifier.

\textbf{macro\_type} there are only two kinds of macros - \verb|MACRO_OBJ| which is used to describe object-like macros, and \verb|MACRO_FUNC| which describes function like macrosses. 

\textbf{str} the lexeme associated with this token.

\textbf{first\_arg} first\_arg param

\textbf{line 1325} the token to this was already allocated and added prior to the same index via \verb|tok_alloc_new()|

\end{tcc_desc}

\section{Finding a define}
\begin{verbatim}
ST_INLN Sym *define_find(int v)
{
    v -= TOK_IDENT;
    if ((unsigned)v >= (unsigned)(tok_ident - TOK_IDENT))
        return NULL;
    return table_ident[v]->sym_define;
}
\end{verbatim}


\section{Defining Object-Like Macros}

the tokenizer sees the '\#' character and determines that a preprocessing operation would be happening next. the tokenizer gets called for the second time to extract the next few characters - "define" in this case and a hash table
index is computed using the characters. the symbol entry for "define" is searched in the hash table and naturally, it will be found, so a new token will not be allocated, but instead, a reference to the symbol for "define" will be returned.

the global tok variable is set to the token id for "define" so it can be used by other parts of the compiler.

the tokenizer is called for the third time to extract the macro identifier. A new token symbol is created for this identifer and it is added to the hash table.

the global tok variable is set to the token id for this indentifier now.

now it is time to parse the object-like macro. the function parse\_define() is called to do this.

the tokenizer is called a fourth time to extract the value now. when tok is TOK\_PPNUM, the global variable tokc holds the token value and the global tok holds TOK\_PPNUM . TOK\_PPNUM is obviously not a space and the function tok\_str\_add2() to be called so that token id, length of the toke, and value of the token will be appended to the global token string tokstr\_buf.

tok\_str\_add2() utilizes the global tok value to do its operation and when it sees that it is looking at at TOK\_PPNUM token, it will consult the global CValue to extract the actual value.

a last call to the tokenizer will simply update the internal tracking of the tokenizer to point to next line.




% preprocessing-token: pp-number

% pp-number:
%       digit
%       . digit
%       pp-number digit
%       pp-number identifier-nondigit
%       pp-number e sign
%       pp-number E sign
%       pp-number p sign
%       pp-number P sign
%       pp-number .

% defining simple ints
% simple ints fall under the digit production rule 

% c => peeked character 
% t => former character
%
% t is already checked to be valid. (initially it is checked to be a number)
% now check if the peeked character is also valid.
% 
% (!(         (isidnum_table[c - CH_EOF] & (IS_ID|IS_NUM)) // check if peeked char is a number. => digit production rule
%          || c == '.'                                     // if peeked char is not a number, is it a decimal sign? => pp-number . production rule


%             // if peeked char is a sign '+' or '-', 
%             // then the former must be 'e' or 'E'       +e or +E => e sign or E sign production rule
%             // or 'p' or 'P'                            +p or +P => p sign or P sign production rule
%          || ( (c == '+' || c == '-') && (     (    (t == 'e' || t == 'E')
%                                                 && !(parse_flags & PARSE_FLAG_ASM_FILE && ((char*)tokcstr.data)[0] == '0' && toup(((char*)tokcstr.data)[1]) == 'X')
%                                               )
%                                            || t == 'p' || t == 'P'
%                                         )
%             )
%    )
% )


\begin{verbatim}
1478 ST_FUNC void parse_define(void)
1479 {
1480     Sym *s, *first, **ps;
1481     int v, t, varg, is_vaargs, spc;
1482     int saved_parse_flags = parse_flags;
1483
1484     v = tok;
1485     if (v < TOK_IDENT || v == TOK_DEFINED)
1486         tcc_error("invalid macro name '%s'", get_tok_str(tok, &tokc));
1487     /* XXX: should check if same macro (ANSI) */
1488     first = NULL;
1489     t = MACRO_OBJ;
....    
1494     parse_flags = ((parse_flags & ~PARSE_FLAG_ASM_FILE) | PARSE_FLAG_SPACES);
....    
1496     next_nomacro_spc();
1497     if (tok == '(') {
....
1527     }
1528
1529     tokstr_buf.len = 0;
1530     spc = 2;
1531     parse_flags |= PARSE_FLAG_ACCEPT_STRAYS | PARSE_FLAG_SPACES | PARSE_FLAG_LINEFEED;
....
1537     while (tok != TOK_LINEFEED && tok != TOK_EOF) {
1538         /* remove spaces around ## and after '#' */
1539         if (TOK_TWOSHARPS == tok) {
1540             if (2 == spc)
1541                 goto bad_twosharp;
1542             if (1 == spc)
1543                 --tokstr_buf.len;
1544             spc = 3;
1545             tok = TOK_PPJOIN;
1546         } else if ('#' == tok) {
1547             spc = 4;
1548         } else if (check_space(tok, &spc)) {
1549             goto skip;
1550         }
1551         tok_str_add2(&tokstr_buf, tok, &tokc);
1552     skip:
1553         next_nomacro_spc();
1554     }
1555
1556     parse_flags = saved_parse_flags;
1557     if (spc == 1)
1558         --tokstr_buf.len; /* remove trailing space */
1559     tok_str_add(&tokstr_buf, 0);
1560     if (3 == spc)
1561 bad_twosharp:
1562         tcc_error("'##' cannot appear at either end of macro");
1563     define_push(v, t, tok_str_dup(&tokstr_buf), first);
1564 }
\end{verbatim}



\section{Defining Function-Like Macros}

\section{Undefining}

\section{Including Files}

\section{Error}

\section{Pragmas}

\section{Conditional Preprocessing}

\subsubsection{Ifdef}
\subsubsection{Ifndef}
\subsubsection{Elif}

\subsubsection{Ending a Conditional Block}




\part{Code Generation}

\part{Auxilliary Modules}

\chapter{Options Processing}

\chapter{Makefile}

\end{document}                          % The required last line

