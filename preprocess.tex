
\chapter{Preprocessing}

say something about the Preprocessing stage of compilation here.

\section{The define stack}

explain what the define stack is.


The function \verb|define_push()| is used to add to the define stack. It does this by first associating a lexeme to a symbol object, then associating this symbol to a token. This symbol object can only be one of two kinds of macro types \verb|MACRO_OBJ| or \verb|MACRO_FUNC|.


\begin{verbatim}
1317 ST_INLN void define_push(int v, int macro_type, int *str, Sym *first_arg)
1318 {
1319     Sym *s, *o;
1320
1321     o = define_find(v);
1322     s = sym_push2(&define_stack, v, macro_type, 0);
1323     s->d = str;
1324     s->next = first_arg;
1325     table_ident[v - TOK_IDENT]->sym_define = s;
1326 
1327     if (o && !macro_is_equal(o->d, s->d))
1328         tcc_warning("%s redefined", get_tok_str(v, NULL));
1329 }
\end{verbatim}
\begin{tcc_desc}
\textbf{v} \verb|v| is the token for the identifier.

\textbf{macro\_type} there are only two kinds of macros - \verb|MACRO_OBJ| which is used to describe object-like macros, and \verb|MACRO_FUNC| which describes function like macrosses. 

\textbf{str} the lexeme associated with this token.

\textbf{first\_arg} first\_arg param

\textbf{line 1325} the token to this was already allocated and added prior to the same index via \verb|tok_alloc_new()|

\end{tcc_desc}

\section{Finding a define}
\begin{verbatim}
ST_INLN Sym *define_find(int v)
{
    v -= TOK_IDENT;
    if ((unsigned)v >= (unsigned)(tok_ident - TOK_IDENT))
        return NULL;
    return table_ident[v]->sym_define;
}
\end{verbatim}


\section{Defining Object-Like Macros}


\begin{enumerate}
\item The tokenizer sees the \emph{\#} character and determines that a preprocessing operation would be happening next. The global \verb|tok| variable is set to \emph{\#}.

\item The tokenizer gets called for the second time to extract the next few characters \emph{define}. A hash table index is computed using these characters.

\item The symbol entry for \emph{define} is searched in the hash table and naturally, it will be found, so a new \verb|TokenSym| will not be allocated, but instead, a reference to the \verb|TokenSym| for \emph{define} will be returned.

\item \verb|TokenSym| for each identifier contains references to all the information you ever need to deal with that identifier. Infact, each \verb|TokenSym| also holds the tok number dynamically assigned for this particular token. The global \verb|tok| variable is set to the token number assigned to \emph{define} so it can be used by other parts of the compiler.

\item The tokenizer is called for the third time to extract the macro identifier - the string after \emph{define}. The preprocessor will determine if this identifier has already been registered or not. If not, a new \verb|TokenSym| is created for this identifer and it is added to the hash table. Remember that every new \verb|TokenSym| is already assigned a token number too and this is stored on the \verb|TokenSym| created for this identifier.

\item The global \verb|tok| variable is set to the token number for this indentifier. The lexer at this point has now extracted three tokens from a single preprocessing line. The \emph{\#} symbol, the \emph{define} string, and third, the \emph{identifier}.

\item Now it is time to process what is referred to as \emph{pp-tokens} or preprocessing tokens. The function \verb|parse_define()| is called to do this. \emph{pp-tokens} can be things like simple integers, string literals, even floating points and even numbers in scientific notation.

\item The tokenizer is called a fourth time to extract the next token now. Depending on what it encounters the global \verb|tok| value is set again to reflect the next token. For example, if a string literal is detected - e.g. a string enclosed in double quotes, \verb|tok| is set to \verb|PP_STR|. If a plain-old integer number is detected that \verb|tok| gets set to \verb|TOK_PPNUM|.

\item Ultimately, a new define symbol is created and pushed to the define stack via \verb|define_push()|. This new define symbol is linked properly to the token for the identifer as well.


\end{enumerate}



% preprocessing-token: pp-number


% the . digit production rule is handled by the case '.' ....


when the macro is just a simple int value:
        set the global variable to hold a constant value \verb|tokc|
        tokc.size = \# of digits
        tokc.data = simple int in string form
        tok = TOK\_PPNUM


\subsubsection{Transferring a Constant Value to a string}

The global variable \verb|tokc| holds a valid constant value when the global \verb|tok| value is \verb|TOK_PPNUM|. Outlined below are the specific lines of \verb|tok_str_add2()| function handling a \verb|TOK_PPNUM| token.

\begin{verbatim}
1145 static void tok_str_add2(TokenString *s, int t, CValue *cv)
1146 {
1147     int len, *str;
1148 
1149     len = s->lastlen = s->len;
1150     str = s->str;
1151 
1152     /* allocate space for worst case */
1153     if (len + TOK_MAX_SIZE >= s->allocated_len)
1154         str = tok_str_realloc(s, len + TOK_MAX_SIZE + 1);
1155     str[len++] = t;
         ...
1169     case TOK_PPNUM:
1170     case TOK_PPSTR:
1171     case TOK_STR:
1172     case TOK_LSTR:
1173         {
1174             /* Insert the string into the int array. */
1175             size_t nb_words =
1176                 1 + (cv->str.size + sizeof(int) - 1) / sizeof(int);
1177             if (len + nb_words >= s->allocated_len)
1178                 str = tok_str_realloc(s, len + nb_words + 1);
1179             str[len] = cv->str.size;
1180             memcpy(&str[len + 1], cv->str.data, cv->str.size);
1181             len += nb_words;
1182         }
1183         break;
1212     default:
1213         break;
1214     }
1215     s->len = len;
1216 }
\end{verbatim}

\begin{tcc_desc}
        \textbf{line 1149} s->len is counted in terms of words, not byte.

        \textbf{line 1149-1150} local references to TokenString object.

        \textbf{line 1153-1154} resize (enlarge) the TokenString object buffer so it can fit more data.

        \textbf{line 1155} field 0 when representing a PPNUM holds the token.

        \textbf{line 1169-1172} all of these 4 token types have the same handling.

        \textbf{line 1175-1178} \verb|tok_str_realloc()| allocates in terms of the system integer width (usually 4 bytes, also called a "word"). This line computes the number of words necessary to cover the string length. e.g. if string length is 3, then we only need 1 word. If however we have a string length of 5, then it rounds up the word count to 2. Additionally, +1 word is added to be able to save the string length itself into field 1.

        \textbf{line 1179} field 1 is used to store the string length.

        \textbf{line 1180} the actual string gets copied here into field 2.

        \textbf{line 1181 and line 1215} reflect the new length now.
\end{tcc_desc}

It is important to remember that TokenString keeps strings as an array of words (or ints) which is usually 4 bytes. Therefore, the length field inside a TokenString object is not the actual string length. Instead, it reflects the length of the buffer in terms of words.

\begin{verbatim}
1478 ST_FUNC void parse_define(void)
1479 {
1480     Sym *s, *first, **ps;
1481     int v, t, varg, is_vaargs, spc;
1482     int saved_parse_flags = parse_flags;
1483
1484     v = tok;
1485     if (v < TOK_IDENT || v == TOK_DEFINED)
1486         tcc_error("invalid macro name '%s'", get_tok_str(tok, &tokc));
1487     /* XXX: should check if same macro (ANSI) */
1488     first = NULL;
1489     t = MACRO_OBJ;
....    
1494     parse_flags = ((parse_flags & ~PARSE_FLAG_ASM_FILE) | PARSE_FLAG_SPACES);
....    
1496     next_nomacro_spc();
1497     if (tok == '(') {
....
1527     }
1528
1529     tokstr_buf.len = 0;
1530     spc = 2;
1531     parse_flags |= PARSE_FLAG_ACCEPT_STRAYS | PARSE_FLAG_SPACES | PARSE_FLAG_LINEFEED;
....
1537     while (tok != TOK_LINEFEED && tok != TOK_EOF) {
1538         /* remove spaces around ## and after '#' */
1539         if (TOK_TWOSHARPS == tok) {
1540             if (2 == spc)
1541                 goto bad_twosharp;
1542             if (1 == spc)
1543                 --tokstr_buf.len;
1544             spc = 3;
1545             tok = TOK_PPJOIN;
1546         } else if ('#' == tok) {
1547             spc = 4;
1548         } else if (check_space(tok, &spc)) {
1549             goto skip;
1550         }
1551         tok_str_add2(&tokstr_buf, tok, &tokc);
1552     skip:
1553         next_nomacro_spc();
1554     }
1555
1556     parse_flags = saved_parse_flags;
1557     if (spc == 1)
1558         --tokstr_buf.len; /* remove trailing space */
1559     tok_str_add(&tokstr_buf, 0);
1560     if (3 == spc)
1561 bad_twosharp:
1562         tcc_error("'##' cannot appear at either end of macro");
1563     define_push(v, t, tok_str_dup(&tokstr_buf), first);
1564 }
\end{verbatim}



\section{Defining Function-Like Macros}


\section{Finding a define Identifier}

The function \verb|define_find()| returns a reference to the symbol object associated with a token \verb|v|. It is outlined below with explanation.

\begin{verbatim}
ST_INLN Sym *define_find(int v)
{
    v -= TOK_IDENT;
    if ((unsigned)v >= (unsigned)(tok_ident - TOK_IDENT))
        return NULL;
    return table_ident[v]->sym_define;
}
\end{verbatim}

\begin{tcc_desc}
\textbf{line 244} normalize the token in reference to \verb|TOK_IDENT|.

\textbf{line 245-246} The global token counter \verb|tok_ident| is the source for all token numbers. Therefore, a valid token is always less than the global token counter (that is normalized in reference to \verb|TOK_IDENT|) which is what is checked here. Anything greater than, is obviously a non-existent token as far as TinyCC is concerned.

\textbf{line 247} return the define tokensymbol for token \verb|v|.
\end{tcc_desc}


\section{Registering a define macro to the system - The Define Stack}
The define stack where define symbols are pushed as they are discovered. The function to push new macro symbols to the TinyCC define stack is \verb|define_push()|. It is outlined below and explained line by line.

\begin{verbatim}
1317 ST_INLN void define_push(int v, int macro_type, int *str, Sym *first_arg)
1318 {
1319     Sym *s, *o;
1320
1321     o = define_find(v);
1322     s = sym_push2(&define_stack, v, macro_type, 0);
1323     s->d = str;
1324     s->next = first_arg;
1325     table_ident[v - TOK_IDENT]->sym_define = s;
1326
1327     if (o && !macro_is_equal(o->d, s->d))
1328         tcc_warning("%s redefined", get_tok_str(v, NULL));
1329 }
\end{verbatim}

\begin{tcc_desc}
\textbf{line 1321} return a reference to this identifier - if has been defined prior.

\textbf{line 1322-1324} create and push a new symbol now onto the define stack \verb|define_stack|. Initialize this symbol with additional information pertaining to the macro the compiler just parsed.

\textbf{line 1325} store a reference to this actual symbol now inside the token symbol table.

\textbf{line 1327-1328} detect if this particular macro identifier has already been prior. Then, alert user of the fact.
\end{tcc_desc}


\subsection{Explaining the preprocessing tokens}

\subsubsection{TOK\_CCHAR}
\subsubsection{TOK\_LCHAR}
\subsubsection{TOK\_CINT}
\subsubsection{TOK\_CUINT}
\subsubsection{TOK\_CLLONG}
\subsubsection{TOK\_CULLONG}
\subsubsection{TOK\_STR}
\subsubsection{TOK\_LSTR}
\subsubsection{TOK\_CFLOAT}
\subsubsection{TOK\_CDOUBLE}
\subsubsection{TOK\_CLDOUBLE}
\subsubsection{TOK\_PPNUM}

% pp-number:
%       digit
%       . digit
%       pp-number digit
%       pp-number identifier-nondigit
%       pp-number e sign
%       pp-number E sign
%       pp-number p sign
%       pp-number P sign
%       pp-number .

% defining simple ints
% simple ints fall under the digit production rule 

% c => peeked character 
% t => former character
%
% t is already checked to be valid. (initially it is checked to be a number)
% now check if the peeked character is also valid.
% 
% (!(         (isidnum_table[c - CH_EOF] & (IS_ID|IS_NUM)) // check if peeked char is a number. => digit production rule
%          || c == '.'                                     // if peeked char is not a number, is it a decimal sign? => pp-number . production rule


%             // if peeked char is a sign '+' or '-', 
%             // then the former must be 'e' or 'E'       +e or +E => e sign or E sign production rule
%             // or 'p' or 'P'                            +p or +P => p sign or P sign production rule
%          || ( (c == '+' || c == '-') && (     (    (t == 'e' || t == 'E')
%                                                 && !(parse_flags & PARSE_FLAG_ASM_FILE && ((char*)tokcstr.data)[0] == '0' && toup(((char*)tokcstr.data)[1]) == 'X')
%                                               )
%                                            || t == 'p' || t == 'P'
%                                         )
%             )
%    )
% )

The C standard defines what pp-numbers are and is shown below:

\begin{verbatim}
pp-number:
      digit
      . digit
      pp-number digit
      pp-number identifier-nondigit
      pp-number e sign
      pp-number E sign
      pp-number p sign
      pp-number P sign
      pp-number .
\end{verbatim}

Assigned from the above grammar rules, Preprocessing number tokens lexically include all floating and integer constant tokens. The C standard contains the complete list, but to give you an idea, integer constants include the following:

\begin{itemize}
\item decimal constants with integer suffix:
        \begin{itemize}
        \item unsigned suffix (u or U) as in 1000U
        \item long suffix (l or L) as in 1024L
        \item long long suffix (ll or LL) as in 2048LL
        \item any valid combination of unsigned and long suffix, as in 512ULL
        \end{itemize}

\item octal constants: '0' followed by a valid octal number as in 0777

\item hexadecimal constants: "0x" followed by any valid hexadecimal number as in 0x1000
\end{itemize}

This character sequence is extracted and placed into the global \verb|tokc| variable which is supposed to hold the most recently processed constant values. At this point, the supposed constant number value is not treated as a numeric value yet. The lexeme is recorded verbatim to the global \verb|CValue| variable without labelling it as an int or float yet.


\subsubsection{TOK\_PPSTR}
\subsubsection{TOK\_LINENUM}
\subsubsection{TOK\_TWODOTS}






\section{Undefining}

\section{Including Files}

\subsection{The \#include preprocessing directive}

A \verb|#include| directive shall identify a header or source file that can be processed by the implementation.\footnote{ISO/IEC 9899:TC3 Committee Draft — Septermber 7, 2007, 6.10.2.1} Subsequently, the implementation will cause the replacement of that include directive line with the entire contents of the header.

There are two ways to declare the source file for an include directive line. The first one is to use the \verb|<| and \verb|>| delimiters. Declaring the file this way causes the implementation to search for the file in \emph{implementatation-defined} places - e.g. The \verb|/usr/include/| directory is a standard directory where the GCC looks for header files.

The second way to declare the source file is to enclose the source file in double quotes. The file is then searched for in an \emph{implementatation-defined} manner. Usually, this means the compiler will search for the file in the same directory where the source file is located.

\section{Error}

\section{Pragmas}

\section{Conditional Preprocessing}

\subsubsection{Ifdef}
\subsubsection{Ifndef}
\subsubsection{Elif}

\subsubsection{Ending a Conditional Block}


