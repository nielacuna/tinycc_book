\chapter{String Management}


\section{The CString Object}

CStrings are TCC implementations of raw C strings (character arrays). Because 
Raw C strings are usually tags or pointers to contiguous regions of memory,
dynamically resizing them (more commonly, enlarging them) is tedious. A new
larger memory block needs to be allocated and the old string copied to this new
block. Care must be taken so that the new larger block of memory is large enough
for the new additions to the string to be added safely. Also, the old memory
address of the string must be \verb|free()|'d back to the platform so it can be reused.

It is declared as follows in <tcc.h>.
\begin{verbatim}
425 typedef struct CString {
426     int size;
427     void *data;
428     int size_allocated;
429 } CString;
\end{verbatim}

\begin{tcc_desc}
\textbf{size} The current memory usage of the CString object - which is also the current string length not counting the terminating NULL. This must always be less than or equal to the \verb|size_allocated| member field.

\textbf{data} This is a pointer to the memory region containing the raw C string.

\textbf{size\_allocated} Describes the maximum growable size of the CString. This is not the actual size of the string, that is what the \verb|size| member field is for. The CString is usually reallocated to a bigger memory when it is determined that future concatenations to the string will overflow \verb|size_allocated| soon.
\end{tcc_desc}

\subsection{Creating CStrings}

\begin{verbatim}
370 ST_FUNC void cstr_new(CString *cstr)
371 {
372     memset(cstr, 0, sizeof(CString));
373 }
\end{verbatim}
CString objects start their life as an empty string. They don't hold any strings at all and subsequently have a length of 0. This function ensures the CString is in a known state before even the first string is assigned to it.

\subsection{Destroying CStrings}

\begin{verbatim}
ST_FUNC void cstr_free(CString *cstr)
{
    tcc_free(cstr->data);
    cstr_new(cstr);
}
\end{verbatim}
free string and reset it to NULL.

\subsection{Resetting CStrings}

*** ST\_FUNC void cstr\_reset(CString *cstr)
reset string to empty.

\subsection{Concatenating a single Character}

*** ST\_INLN void cstr\_ccat(CString *cstr, int ch)
add a byte. reallocate memory if needed.

\subsection{Concatenating a string to a CString}

*** ST\_FUNC void cstr\_cat(CString *cstr, const char *str, int len)
add a range of bytes, range is defined by len parameter

\subsection{Formatted Concatenation to CStrings}

*** ST\_FUNC int cstr\_printf(CString *cstr, const char *fmt, ...)
provides a way to continuously append strings to a CString.


ST\_FUNC void parse\_mult\_str (CString *astr, const char *msg)


\section{Other String Handling Patterns}

tcc\_split\_path()



/* copy a string and truncate it. */
ST\_FUNC char *pstrcpy(char *buf, size\_t buf\_size, const char *s)


/* strcat and truncate. */
ST\_FUNC char *pstrcat(char *buf, size\_t buf\_size, const char *s)


ST\_FUNC char *pstrncpy(char *out, const char *in, size\_t num)

