
\chapter{Symbols}
Symbols are TinyCC representations of various entities like variable names, function names and other objects.


\section{The symbol pool}
Symbols are allocated from a symbol pool. And the symbol pool is inside a dynamically enlarged collection of symbol pools. The function \verb|__sym_malloc()| takes care of managing this collection. It is explained further below.

\begin{verbatim}
tccgen.c
40 static Sym *sym_free_first;
41 static void **sym_pools;
42 static int nb_sym_pools;
\end{verbatim}
\begin{tcc_desc}
\textbf{line 40} The next sym object to be given away.

\textbf{line 41} Collection of sym pools array.

\textbf{line 42} Count of pools in the sym pools array.
\end{tcc_desc}

\subsection{Allocating/Enlarging the symbol pool}

\begin{verbatim}
691 static Sym *__sym_malloc(void)
692 {
693     Sym *sym_pool, *sym, *last_sym;
694     int i;
695 
696     sym_pool = tcc_malloc(SYM_POOL_NB * sizeof(Sym));
697     dynarray_add(&sym_pools, &nb_sym_pools, sym_pool);
698
699     last_sym = sym_free_first;
700     sym = sym_pool;
701     for(i = 0; i < SYM_POOL_NB; i++) {
702         sym->next = last_sym;
703         last_sym = sym;
704         sym++;
705     }
706     sym_free_first = last_sym;
707     return last_sym;
708 }
\end{verbatim}

\begin{tcc_desc}
\textbf{line 696} Allocate a pool of symbols.

\textbf{line 697} Add this new pool to the sym pools array.

\textbf{line 699} Save a reference to the last sym on the current sym pool.

\textbf{line 700-705} Link all the symbols in this new sym pool together. This is a backwards pointing singly linked list of free sym objects.

\textbf{line 706-707} Return a reference to the first (actually the last) sym object in this pool.
\end{tcc_desc}

\subsection{Allocating a symbol}
Symbols are allocated from a pre-allocated pool of syms. The next free sym to give out is simply the next sym object in a singly linked list of free syms as shown below inside the \verb|sym_malloc()|.

\begin{verbatim}
710 static inline Sym *sym_malloc(void)
711 {
712     Sym *sym;
713 #ifndef SYM_DEBUG
714     sym = sym_free_first;
715     if (!sym)
716         sym = __sym_malloc();
717     sym_free_first = sym->next;
718     return sym;
719 #else
720     sym = tcc_malloc(sizeof(Sym));
721     return sym;
722 #endif
723 }
\end{verbatim}
\begin{tcc_desc}
\textbf{line 714} Get a reference to the next free sym object in the current sym pool.

\textbf{line 715-716} If the current sym pool has been exhausted, allocate a new pool.

\textbf{line 717} Set up the next sym object to be given out next time a symbol is requested to be allocated.

\textbf{line 718} Return reference to this newly "allocated" symbol.

\textbf{line 720-721} This is the slow and more resource intensive symbol allocation. Allocate symbol objects individually when requested.
\end{tcc_desc}

\section{Freeing a symbol}
This API is never used in TinyCC. Freeing a symbol is fast as it involves simply appending a sym object at the front of the sym pool. 

\begin{verbatim}
725 ST_INLN void sym_free(Sym *sym)
726 {
727 #ifndef SYM_DEBUG
728     sym->next = sym_free_first;
729     sym_free_first = sym;
730 #else
731     tcc_free(sym);
732 #endif
733 }
\end{verbatim}

\begin{tcc_desc}
\textbf{line 728-729} Append \verb|sym| to the front of the free sym pool.

\textbf{line 731} This is the slow and more resource intensive symbol freeing path.
\end{tcc_desc}

\section{Registering Symbols to the Compiler System}

\subsection{Lowlevel push}
\begin{verbatim}
736 ST_FUNC Sym *sym_push2(Sym **ps, int v, int t, int c)
737 {
738     Sym *s;
739
740     s = sym_malloc();
741     memset(s, 0, sizeof *s);
742     s->v = v;
743     s->type.t = t;
744     s->c = c;
745     /* add in stack */
746     s->prev = *ps;
747     *ps = s;
748     return s;
749 }
\end{verbatim}
\begin{tcc_desc}
\textbf{line 740-741} Allocate new zeroed symbol. 

\textbf{line 742} Associate this symbol with a token. The token will hold the name of this symbol.

\textbf{line 743} Give this symbol a type.

\textbf{line 744} Constant or numerical value associated with this symbol.

\textbf{line 745-747} As the comment implies, add this symbol onto chosen stack \verb|ps|.

\textbf{line 748} return a pointer to this newly created symbol.
\end{tcc_desc}

\subsection{Pushing a symbol}

\begin{verbatim}
792 ST_FUNC Sym *sym_push(int v, CType *type, int r, int c)
793 {
794     Sym *s, **ps;
795     TokenSym *ts;
796
797     if (local_stack)
798         ps = &local_stack;
799     else
800         ps = &global_stack;
801     s = sym_push2(ps, v, type->t, c);
802     s->type.ref = type->ref;
803     s->r = r;
804     /* don't record fields or anonymous symbols */
805     /* XXX: simplify */
806     if (!(v & SYM_FIELD) && (v & ~SYM_STRUCT) < SYM_FIRST_ANOM) {
807         /* record symbol in token array */
808         ts = table_ident[(v & ~SYM_STRUCT) - TOK_IDENT];
809         if (v & SYM_STRUCT)
810             ps = &ts->sym_struct;
811         else
812             ps = &ts->sym_identifier;
813         s->prev_tok = *ps;
814         *ps = s;
815         s->sym_scope = local_scope;
816         if (s->prev_tok && sym_scope(s->prev_tok) == s->sym_scope)
817             tcc_error("redeclaration of '%s'",
818                 get_tok_str(v & ~SYM_STRUCT, NULL));
819     }
820     return s;
821 }
\end{verbatim}

\begin{tcc_desc}
\end{tcc_desc}


\section{Popping a symbol}

\section{Finding a symbol}

\begin{verbatim}
ST_INLN Sym *sym_find(int v)
{
    v -= TOK_IDENT;
    if ((unsigned)v >= (unsigned)(tok_ident - TOK_IDENT))
        return NULL;
    return table_ident[v]->sym_identifier;
}
\end{verbatim}

\begin{verbatim}
ST_FUNC Sym *sym_find2(Sym *s, int v)
{
    while (s) {
        if (s->v == v)
            return s;
        else if (s->v == -1)
            return NULL;
        s = s->prev;
    }
    return NULL;
}
\end{verbatim}

\begin{verbatim}
736 ST_FUNC Sym *sym_push2(Sym **ps, int v, int t, int c)
737 {
738     Sym *s;
739 
740     s = sym_malloc();
741     memset(s, 0, sizeof *s);
742     s->v = v;
743     s->type.t = t;
744     s->c = c;
745     /* add in stack */
746     s->prev = *ps;
747     *ps = s;
748     return s;
749 }
\end{verbatim}

\section{Symbol Stacks}

\subsection{The define\_stack}

\subsection{The global\_stack}

\subsection{The local\_stack}

\subsection{The global\_label\_stack}

\section{The local\_label\_stack}

%\subsection{The ifdef Stack}

%\subsection{The pack Stack}


